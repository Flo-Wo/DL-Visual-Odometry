\section{Current and further work}

\subsection{Acquire more data}
\begin{frame}{Acquire more data and evaluation of the test video}
	\textbf{Method}
	\begin{itemize}
%		\item using simple methods it is still possible to create more data in good quality
		\item video producing and velocity detection with common apps
		\setbeamertemplate{itemize items}[circle]
		\begin{itemize}
		\item \textit{open street maps}: GPS tracking (\texttt{.gpx}-files)
		\item \textit{open camera}: dashcam footage
		\end{itemize}				
%		(\textit{open camera} and \textit{open street maps})
%		\item using GPS tracking for velocity detection (exportable as \texttt{.gpx}-file) and extrapolation
%		\item using the time stamps and interpolation both files can be combined to one data set
	\end{itemize}
	\textbf{Advantages and Issues}
	\begin{itemize}
		\item very easy method to create a lot of data (if a car is available)
		\item velocity has some uncertainties (needs to be extrapolated to cover all frames; uncertainties of GPS vs. car sensors)
		\item frame rate differs a little from original data (24 fps vs 20 fps)
	\end{itemize}
%	\textbf{Evaluation with the test video}
%	\begin{itemize}
%		\item rough check how model performs on unseen video with the same parameters
%		\item no labels available so we can only compare key situations (stops, highway, ...)
%		\item some models achieve at least a qualitative accordance with the video (hier noch ein Bild)
%		\item best networks: altered original network and 
%	\end{itemize}
\end{frame}

%\begin{frame}{Evaluation with the test video}
%	\begin{itemize}
%		\item rough check how model performs on unseen video with the same parameters
%		\item no labels available so we can only compare key situations (stops, highway, ...)
%		\item some models achieve at least a qualitative accordance with the video (hier noch ein Bild)
%		\item best networks: altered original network and 
%	\end{itemize}
%\end{frame}

\subsection{Additional noise}
\begin{frame}{Contrast and brightness augmentation}
\begin{itemize}
\item Additional noise to frames \textbf{before} calculating the flow field.
\item Change the brightness and contrast of an image via
\begin{align*}
\text{frame}_{\mathrm{augmented}}(i,j) = \alpha(i,j) \cdot \text{frame}(i,j) + \beta(i,j)
\end{align*}
with functions $\alpha$ (contrast: $>1$ increase, $<1$ decrease) and $\beta$ (brightness).\\
To get some noise into the frames, we used
\begin{align*}
\alpha &\sim \mathcal{U}(0,1)+0.35\\
\beta &\sim \mathcal{U}(-5,35),
\end{align*}
where $\mathcal{U}(a,b)$ is the uniform distribution in an interval $[a,b]$ for $a < b$.
\end{itemize}
\end{frame}