\section{Current and further work}
\subsection{Additional noise}
\begin{frame}{Contrast and brightness augmentation}
\begin{itemize}
\item Additional noise to frames \textbf{before} calculating the flow field.
\item Change the brightness and contrast of an image via
\begin{align*}
\text{frame}_{\mathrm{augmented}}(i,j) = \alpha(i,j) \cdot \text{frame}(i,j) + \beta(i,j)
\end{align*}
with functions $\alpha$ (contrast: $>1$ increase, $<1$ decrease) and $\beta$ (brightness).\\
To get some noise into the frames, we used
\begin{align*}
\alpha &\sim \mathcal{U}(0,1)+0.35\\
\beta &\sim \mathcal{U}(-5,35),
\end{align*}
where $\mathcal{U}(a,b)$ is the uniform distribution in an interval $[a,b]$ for $a < b$.
\end{itemize}
\end{frame}

\begin{frame}{Acquire more Data}
\begin{itemize}
	\item use mobile to create a video and track the velocity via GPS
	\item test run already worked
	\item data does not compare to the rest due to completely different brightness and angle
	\item but it would allow to create a greater data set with more training and labeld test data
\end{itemize}´
\end{frame}

\begin{frame}{Usage in real application}
	\begin{itemize}
		\item if we have good model we have written method that reads video and predicts velocity
		\item we reach 30 fps which is faster than the 20 fps of the video so life prediction would be possible
	\end{itemize}
\end{frame}

\begin{frame}{Evaluation with the test video}
	\begin{itemize}
		\item rough check how model performs on unseen video with the same parameters
		\item no labels available so we can only compare key situations (stops, highway, ...)
		\item some models achieve at least a qualitative accordance with the video (hier noch ein Bild)
		\item best networks: altered original network and 
	\end{itemize}
\end{frame}