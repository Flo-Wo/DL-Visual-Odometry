\usepackage{amsthm}
\usepackage{animate}
\usepackage{mathtools}
\usepackage{physics}
\usepackage{calligra}
\usepackage{csquotes}
\usepackage{tensor}
\usepackage[thicklines]{cancel}
\usepackage{tcolorbox}
\usepackage{pstricks}
\usepackage{ulem}
%\usepackage[ngerman]{babel} % Language hyphenation and typographical rules
%\usepackage[backend=biber, style=numeric]{biblatex}

\setbeamersize{text margin left=0.5cm,text margin right=0.5cm}


\DeclareMathAlphabet{\mathcalligra}{T1}{calligra}{m}{n}
\DeclareFontShape{T1}{calligra}{m}{n}{<->s*[2.2]callig15}{}
\newcommand{\scriptr}{\mathcalligra{r}\,}
\newcommand{\boldscriptr}{\pmb{\mathcalligra{r}}\,}
\def\rc{\scriptr}
\def\brc{\boldscriptr}
\def\hrc{\hat\brc}
\newcommand{\ie}{\emph{i.e.}} %id est
\newcommand{\eg}{\emph{e.g.}} %exempli gratia
\newcommand{\rtd}[1]{\ensuremath{\left\lfloor #1 \right\rfloor}}
\newcommand{\dirac}[1]{\ensuremath{\delta \left( #1 \right)}}
\newcommand{\diract}[1]{\ensuremath{\delta^3 \left( #1 \right)}}
\newcommand{\e}{\ensuremath{\epsilon_0}}
\newcommand{\m}{\ensuremath{\mu_0}}
\newcommand{\V}{\ensuremath{\mathcal{V}}}
\newcommand{\prnt}[1]{\ensuremath{\left(#1\right)}} %parentheses
\newcommand{\colch}[1]{\ensuremath{\left[#1\right]}} %square brackets
\newcommand{\chave}[1]{\ensuremath{\left\{#1\right\}}}  %curly brackets

\useoutertheme{infolines}
\useinnertheme{rectangles}
\usefonttheme{professionalfonts}


%\definecolor{blue}{RGB}{0, 169, 224}
\definecolor{blue}{RGB}{0, 130, 224}
\definecolor{gray}{HTML}{ffffff}
\definecolor{yellow}{HTML}{f0be52}
\definecolor{lightblue}{RGB}{89, 199, 254}

\renewcommand{\CancelColor}{\color{blue}}

\makeatletter
\newcommand{\mybox}[1]{%
  \setbox0=\hbox{#1}%
  \setlength{\@tempdima}{\dimexpr\wd0+13pt}%
  \begin{tcolorbox}[colback=blue,colframe=blue,boxrule=0.5pt,arc=4pt,
      left=6pt,right=6pt,top=6pt,bottom=6pt,boxsep=0pt,width=\@tempdima]
    \textcolor{black}{#1}
  \end{tcolorbox}
}
\makeatother

\usecolortheme[named=blue]{structure}
\usecolortheme{sidebartab}
\usecolortheme{orchid}
\usecolortheme{whale}
\setbeamercolor{alerted text}{fg=yellow}
\setbeamercolor{block title alerted}{bg=alerted text.fg!90!black}
\setbeamercolor{block title example}{bg=lightblue!60!black}
\setbeamercolor{background canvas}{bg=gray}
\setbeamercolor{normal text}{bg=gray,fg=black}

\setbeamertemplate{footline}
        {
      \leavevmode%
      \hbox{%
      \begin{beamercolorbox}[wd=.333333\paperwidth,ht=2.25ex,dp=1ex,center]{author in head/foot}%
        \usebeamerfont{author in head/foot}\insertshortauthor~~(\insertshortinstitute)
      \end{beamercolorbox}%
      \begin{beamercolorbox}[wd=.333333\paperwidth,ht=2.25ex,dp=1ex,center]{title in head/foot}%
        \usebeamerfont{title in head/foot}\insertshorttitle
      \end{beamercolorbox}%
      \begin{beamercolorbox}[wd=.333333\paperwidth,ht=2.25ex,dp=1ex,center]{date in head/foot}%
        \usebeamerfont{date in head/foot}\insertshortdate{}%\hspace*{2em}

    %#turning the next line into a comment, erases the frame numbers
        %\insertframenumber{} / \inserttotalframenumber\hspace*{2ex} 

      \end{beamercolorbox}}%
      \vskip0pt%
    }


\setbeamertemplate{blocks}[rectangle]
\setbeamercovered{dynamic}

\setbeamertemplate{section page}
{
	\begin{centering}
		\begin{beamercolorbox}[sep=27pt,center]{part title}
			\usebeamerfont{section title}\insertsection\par
			\usebeamerfont{subsection title}\insertsubsection\par
		\end{beamercolorbox}
	\end{centering}
}

%\setbeamertemplate{subsection page}
%{
%	\begin{centering}
%		\begin{beamercolorbox}[sep=12pt,center]{part title}
%			\usebeamerfont{subsection title}\insertsubsection\par
%		\end{beamercolorbox}
%	\end{centering}
%}

\newcommand{\hlight}[1]{\colorbox{violet!50}{#1}}
\newcommand{\hlighta}[1]{\colorbox{red!50}{#1}}

\settowidth{\leftmargini}{\usebeamertemplate{itemize item}}
\addtolength{\leftmargini}{\labelsep}

\usepackage{ragged2e}
\usepackage{cases}

\mathcode`\*="8000
{\catcode`\*\active\gdef*{\cdot}}

\usepackage[]{siunitx}

\sisetup{per-mode=fraction, locale = US, expproduct = cdot }
\sisetup{locale = UK}							%	Automatische Einstellung der Ausgabe für bestimmte Regionen (UK, US, DE, FR, ZA)
\sisetup{
	per-mode=fraction,
	fraction-function=\tfrac
}
\DeclareSIUnit\Hy{H}
\DeclareSIUnit\hPa{hPa}
\DeclareSIUnit\Torr{Torr}
\DeclareSIUnit\T{T}
% FLORIAN
\usepackage{graphics}
\def\R{{\mathbb{R}}}
\def\N{{\mathbb{N}}}
\def\C{{\mathbb{C}}}
\def\K{{\mathbb{K}}}
\def\Q{{\mathbb{Q}}}
\def\Z{{\mathbb{Z}}}
\def\O{{\mathcal{O}}}
\def\Pot{{\mathcal{P}}}
\def\P{{\mathbb{P}}}
\def\D{{\mathcal{D}}}
\def\B{{\mathcal{B}}}
\def\U{{\mathcal{U}}}
\def\F{{\mathcal{F}}}
\def\M{{\mathcal{M}}}
\def\E{{\mathbb{E}}}
\def\EEE{{\mathcal{E}}}
\def\Cov{{\mathbb{C}\mathrm{ov}}}
\def\Var{{\mathbb{V}\mathrm{ar}}}
\def\V{{\mathbb{V}}}
\def\ND{{\mathcal{N}}}
\def\essup{{\mathrm{ess \;sup}}}

\usepackage{multirow}
\usepackage{booktabs}
\usepackage{subfigure}