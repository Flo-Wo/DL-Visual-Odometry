\section{Motivation and initial Dataset}


\begin{frame}{The \enquote{comma ai speed challenge}\footnote{https://github.com/commaai/speedchallenge}}
	\textbf{Motivation}
	\begin{itemize}
		\item Autonomous driving is currently one of the most prominent problems in machine learning
		\item But quite hard to set up on a desktop pc
		\item Predicting a vehicles velocity from video footage is a related, but also a much more simplified task
	\end{itemize}
	\pause
	\textbf{Initial Dataset:}
	\begin{itemize}
		\item Training video with 20400 frames (20 fps)
		\item Data file with velocity of the car at each frame
		\item Test video with 10798 frames (20 fps)
	\end{itemize}
	\pause
	\textbf{Evaluation:}
	\begin{itemize}
		\item The mean squared error (MSE) is used to measure performance
		\begin{align*}
			\mathcal{L} = \sum_i (p(x_i) - y_i)^2
		\end{align*}
	\end{itemize}
\end{frame}

\begin{comment}

\begin{frame}{The \enquote{comma ai speed challenge}\footnote{https://github.com/commaai/speedchallenge}}
\textbf{Motivation}
\begin{itemize}
\item HERE ARE SOME MOTIVATIONAL WORDS NEEDED
\end{itemize}
\textbf{Data collection:}
\begin{itemize}
\item \enquote{comma ai speed challenge} provides two videos:
\setbeamertemplate{itemize items}[circle]
\begin{itemize}
\item Train video: 24000 frames, shoot at 20 frames per second, including ground truths
\item Test video: 10798 frames, shoot at 20 frames per second, no ground truths, used to applications
\end{itemize}
\item Split train video after 80$\%$ with hard cut off (ability the generalize), to get train and test datasets
\end{itemize}
\textbf{Initial assumptions}
\begin{itemize}
\item<+-> Use mean squared error (MSE) as a performance measure
\item<+-> How to evaluate a prediction? Assumptions:
\setbeamertemplate{itemize items}[circle]
\begin{itemize}
\item MSE $\leq 10$: good
\item MSE $\leq 5$: better
\item MSE $\leq 3$: correct
\end{itemize}
\end{itemize}
\end{frame}

\end{comment}