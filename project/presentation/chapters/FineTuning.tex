\section{Fine-tuning of the model}
\subsection{Batch Normalization, Dropout layers, activation function and pooling}

\begin{frame}{Batch Normalization, Dropout layers, activation function and pooling}
\begin{itemize}
\item Batch normalization to speed up the training \cite{BatchNorm2015}
\item Initial activation function: $\mathrm{ReLu}: \mathbb{R} \to \mathbb{R}_0^+, x \mapsto \max\{0,x\}$, still MSE over 15 on the testing set\\
$\Rightarrow$ Overfitting problems
\item Found paper about dropout layers \cite{Dropout2014} to reduce overfit, build in one with dropout probability $p=0.5$
\item Solve problems of dead neurons using
\begin{align*}
\mathrm{leakyReLU} : \mathbb{R} \to \mathbb{R}, x \mapsto \begin{cases}
x, x \geq 0\\
c \cdot x, x <0
\end{cases}
\end{align*}
with $c = 0.01$, MSE of around 12 on the testing set
\end{itemize}
\end{frame}
\subsection{Problems and possible solutions}
\begin{frame}{Problems}
We identified three possible problems for poor results
\begin{enumerate}
\item Too complex model, as initially used for autonomous driving or insufficient amount of information put into the model
\item Problems with different brightnesses/illumination changes in the frames, therefore unstable calculations of the optical flow
\item Too ambiguous splitting, as the training and testing datasets represent totally different road traffic scenarios in the road 
traffic
\end{enumerate}
IMAGE OF THE SPEED DISTRIBUTION
\end{frame}
\begin{frame}{Possible solutions}
We found
\begin{itemize}
\item askdflo
\end{itemize}
\end{frame}
